% Please see the licence.txt file that accompanied this software.

\documentclass[]{scrreprt}

\usepackage[margin=2cm]{geometry}

\usepackage{verbatim}
\usepackage{amsmath}
%\usepackage{amsthm}
\usepackage{amsfonts}
\usepackage[colorlinks=true,linkcolor=black]{hyperref}
\usepackage{makeidx}
\usepackage{algorithm}
\usepackage{algorithmic}
\usepackage{tikz}
\usetikzlibrary{fit,backgrounds,matrix}
\usepackage[amsmath,hyperref,standard,thmmarks]{ntheorem}

\newcommand{\Matlab}{Matlab\textsuperscript{\textregistered~}}

\title{HMMdotEM User Guide}
\author{Nikola Karamanov}
\date{}

% \usepackage{fancyhdr}
% \pagestyle{fancy}
% \rfoot{Copyright Nikola Karamanov}

\begin{document}

\maketitle
\abstract{HMMdotEM is a fast, feature-full, extendable package
for training general discrete-state hidden Markov models
using the EM algorithm. All you need to have is a function that computes the
conditional log-probability for a set of data points given each discrete
state's parameters. Any other information that you have, such as an analytic
solution in the M-Step can be easily added to the existing code. This guide provides help for using
these features. Currently the code has only a \Matlab interface.}

\section*{License}
\verbatiminput{../license.txt}

\tableofcontents
\chapter{Installation} \label{ch:Install}
\section{Quick Start}
This section assumes you have extracted a folder called ``HMMdotEM'' from the
``HMMdotEM.zip'' file you downloaded.
\subsection{Using startup.m}
\begin{enumerate}
	\item Inside \Matlab, switch to your HMMdotEM directory and run 
\begin{verbatim}
>> startup
\end{verbatim}
Now you can switch to any other directory (and the code will still be
accessible).\\
Note: If you follow this installation procedure you must follow this step every
time.
	\item To setup the package, mex files, other optimizations and optional
components run
\begin{verbatim}
>> HMMdotEM.configure();
\end{verbatim}
Note: You only need to configure the project once for each architecture you are
using.
\end{enumerate}
\subsection{Configuring from another directory.}
\begin{enumerate}
  \item Open the file HMMdotEM.m in the ``config''
folder and change the line:\\{\it PATH = HMMdotEM.deducehmmdotempath;}\\
(If you are wondering what you should change it to, read the instructions above
that line). 
  \item Now add the ``config'' folder to your \Matlab path. (In the
  lower left corner click ``Start'', then ``Desktop Tools'', then ``Path'',
  then click on the ``Add folder'' button and find the folder)
  \item Run the following (This might compile some files):
  \begin{verbatim}>> HMMdotEM.install();\end{verbatim}
  \item Now any time you startup \Matlab, from any directory, you can run
  \begin{verbatim}>> HMMdotEM.startup(); \end{verbatim}
  to use the package.
  \item If you have any problems, try redoing these steps.
\end{enumerate}

\section{Expert: How to personalize the code to your workflow}
There is a file called ``HMMdotEM.m'' in the ``config'' folder. It
defines an object class and contains code for setting up paths and providing
information about the package. There are also methods there that allow you to
recompile files and administer the package in general.

\chapter{Features worth acknowledging}
\section{Any Conditional Likelihood}
You have the ability to reuse the inference in the E-step for any conditional
likelihood that you may think of. There is even a generic and customizable
M-step solver that you can use if no analytic solution exists for your M-step.

\section{Multiple Segments} There is the
ability to train simultaneously on multiple segments/chunks of data.
This is done by using a herein coined notion
of ``Param-Group'' of HMM's. Param-Groups can be constructed from an instance of
a singular HMM and then be passed to the relevant EM functions without any
difference in code.

\section{Efficiency}
Speed and memory have been optimized for. The code is very conservative in
memory, it clears unneeded quantities the moment they are not needed anymore.
C mex files are used in the E-step for speed. If you are having
problems with the compilation or otherwise you can avoid using them
and fall back on slower but trustworthy \Matlab code.

\section{Help and Documentation}
This documentation is contained in the ``doc'' folder.
Various other help can be found in the source's comments.

\chapter{Developing your model}

\section{Object-Orientation}
Recent versions of \Matlab have support for object-orientation.\footnote{This
has existed before, but was based on a non-elegant use of struct, while the
most recent versions handle classes and inheritance in a clean way.}
This package uses object-orientation to facilitate code-reuse and
abstraction. You are not require you to understand the details of object
orientation, but a few things will help you get started faster. Here they are.
\subsection{Matlab Classes}
\Matlab classes get defined in ``.m'' files. Here is an example:
\verbatiminput{Example.m}
For more information see the \Matlab documentation of object orientation.
\subsection{Folders}
Optionally ``.m'' files containing classes can be placed in separate folders
with corresponding names. For ``Example.m'' above this folder would be called
``@Example'', and it can contain other files (see the \Matlab documentation for
details).
\subsection{Inheritance}
Inheritance lets you use alter/add methods/properties of one class to your
liking, while still using all of the other non-altered methods/properties.
Here is an example:
\verbatiminput{AnotherExample.m}

\section{The Matlab Class for your model}
\subsection{Gaussian Mixture Conditional Likelihood}
This package has built-in Gaussian-Mixture HMM in the ``src'' folder. See the
file ``TESTselection.m'' in the ``test" folder for examples on how you use it.
\subsection{Other HMM's}
This package is designed so you don't have to worry about the details of EM or
its implementation. However you will probably want to define something other
than the Gaussian Mixture HMM. To do this you will need to implement (read to
the end of this section to understand everything before you begin):
\begin{itemize}
    \item(Required Constructor) A constructor for a \Matlab class instance
    representing your model.
    \item(Required Conditional Likelihood) A conditional likelihood function of
    your data given the hidden states.
    \item(Optional M-step solver) The function that solves the M step of the EM
    algorithm for your type of conditional likelihood. This is optional because
    there is a built-in general optimizer for the M-step, but that should not be expected
    to give you good results, unless you understand how to tweak it.
    
\end{itemize}
Refer to ``@GaussMixHMM'' and ``HMMTemplate.m'' in the
``src'' folder for example code. Suppose you want to define a new model with a
Exponential($\lambda_h$) conditional distribution, where h is the index of the
hidden state. Then you would copy the contents ``HMMTemplate.m'' to
``Exponential.m" and alter them in accordance with the instructions in the
comments and relying on the example in ``GaussMixHMM.m'' for guidance. A
very basic understand of HMM's will be required.

\section{Details of Implementation}
The methods and properties, as well as the logic of the package are described in
corresponding files in the ``src'' folder.

\section{WARNINGS TO DEVELOPER}
\subsection{Breaking the code}
IT IS STRONGLY ADVISED THAT YOU {\bf DO NOT} ALTER ANY FILES IN THE ORIGINAL
PACKAGE. Doing otherwise may break some of the logic that makes the code run
efficiently or correctly. You should not have to do anything other than define
methods explicitly presented in ``HMMTemplate.m'' whilst following the
instructions therein very carefully.\footnote{This warning is mostly a scare
tactic, you can obviously find things in the main files that are safe to alter.
The maintained view is that you should never need to do this.}

\chapter{Usage}
\section{Workflow}
It is advised to put all new model files in the ``src'' folder and any
scripts in the ``test'' folder. A typical workflow is the
following\footnote{Make sure you have installed the package (see chapter \ref{ch:Install})}:
\begin{verbatim}
(Start Matlab and move to HMMdotEM directory)
>> cd test % Best place to run anything
>> TESTselection(); % Run something (in this case the built-in Gaussian test)
>> restart; % useful, sometimes Matlab has stale object instances hanging around
(Now in "test" folder again, ready to run next script)
>> somescriptforyourmodel; % Some script you wrote for your model
\end{verbatim}
\end{document}
